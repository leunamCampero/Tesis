\documentclass[12pt]{book}
\usepackage[dvipsnames]{xcolor}
\usepackage{amssymb,latexsym}
\usepackage{graphicx}

%\usepackage[spanish,mexico,es-nolayout]{babel}
\usepackage[utf8]{inputenc}
\usepackage{amsmath}
%\usepackage{amssymb}
\usepackage{amsthm}
%\usepackage{graphicx}
\usepackage{color}
\usepackage{tikz}
\usepackage{tkz-berge}
\usepackage{makeidx}
\usepackage{url}
\usepackage{xspace}
\usepackage{tocbibind}
% ver http://gilmation.com/articles/latex-margins-for-book-binding/
% y http://tex.stackexchange.com/questions/50258/margins-of-book-class
\usepackage[margin=3.5cm]{geometry}
\geometry{bindingoffset=1cm}

%\usepackage{babelbib}

\usetikzlibrary{positioning,shapes,fit,arrows,decorations.pathmorphing}
\definecolor{myblue}{RGB}{56,94,141}


\newtheorem{theorem}{Teorema}[section]
\newtheorem{corollary}[theorem]{Corolario}
\newtheorem{proposition}[theorem]{Proposición}

\theoremstyle{definition}

\newtheorem{definition}[theorem]{Definición}
\newtheorem{notation}[theorem]{Notación}
\newtheorem{example}[theorem]{Ejemplo}
\newtheorem{lemma}[theorem]{Lema}

\newcounter{in}
\newcounter{ini}

\DeclareMathOperator{\Cay}{Cay}
\DeclareMathOperator{\diam}{diam}
\DeclareMathOperator{\Stab}{Stab}
\DeclareMathOperator{\Aut}{Aut}
\DeclareMathOperator{\orb}{Orb}

\newcommand{\GAP}{\textsf{GAP}\xspace}
\newcommand{\GRAPE}{\textsf{GRAPE}\xspace}

\makeindex

\newcommand{\elespacio}{1.4cm}

\begin{document}
\mainmatter 
\begin{titlepage}
  \begin{center}
    \null
    \vspace*{\fill}

    \includegraphics[scale=1.2,bb=55 20 0 0]{escudouaeh.pdf}

    \vspace*{\elespacio}

    \textsc{Universidad Autónoma del Estado de Hidalgo}

    \textsc{Instituto de Ciencias Básicas e Ingeniería}

    \textsc{Área Académica de Matemáticas y Física}

    \vspace*{\elespacio}

    {\Huge\bfseries Título\par}

    \vspace*{\elespacio}

    {\large Tesis que para obtener el título de}

    \vspace*{\elespacio}

    {\Large\textsc{Licenciada en Matemáticas Aplicadas}}

    \vspace*{\elespacio}

    {\large presenta}

    \vspace*{\elespacio}

    {\Huge Alumna}

    \vspace*{\elespacio}

    {\large bajo la dirección de}

    \bigskip

    {\Large Dr.~Rafael Villarroel Flores}

    \bigskip

    {Pachuca, Hidalgo. Junio de 2013.}

    \vspace*{\fill}

  \end{center}
\end{titlepage}

\thispagestyle{empty}
\begin{flushleft}
  {\bfseries\Large Resumen}
\end{flushleft}

En esta tesis se hace blah blah blah blah blah blah blah blah blah
blah blah blah blah blah blah blah blah blah.

\vspace{2cm}

\begin{flushleft}
  {\bfseries\Large Abstract}
\end{flushleft}

In this thesis blah blah blah blah blah blah blah blah blah
blah blah blah blah blah blah blah blah blah.



 \newpage \thispagestyle{empty}

\chapter{Representaciones de grupos}
\label{cha:Representaciones de grupos}
Sea $GL\left(n,\mathbb{C}\right)$ el grupo de todas las matrices no singulares de grado $n$ sobre el campo de los números complejos $\mathbb{C}$. Sea $G$ un grupo. Una representación  (matricial) de $G$ es, por definición, un homomorfismo de $G$ en  $GL\left(n,\mathbb{C}\right)$:\\
$\mathbb(A):a\rightarrow A\left(a\right)$
$\mathbb{A}:$ $a\rightarrow A\left(a\right)$, para todo $a\in G$\\
con\\
$\qquad A\left(ab\right)\left(a\right)\left(b\right)$,\\
$\qquad A\left(1\right)=\mathrm{I}$ (la matriz identidad)\\
$\qquad A\left(a^{-1}\right)=A\left(a\right)^{-1}$ \\
y $n$ es el grado. Se dice que la representación es fiel, si $\mathbb{A}$ es biyectiva.\\~\\
\textbf{Ejemplo 1.1.} El mapeo que a manda cada elemento de $G$ a $1 \in \mathbb{C}$ es una representación de grado 1. Ésta es llamada la representación unitaria de $G$, y es denotada por $1_{G}$.\\~\\
\textbf{Ejemplo 1.2.} Dada una representación $a \rightarrow A\left(a\right)$, el mapeo\\~\\
$a \rightarrow P^{-1}A\left(a\right)P$, para todo $a \in G$\\~\\
se convierte en una representación de $G$ para cualquier matriz $P$ no singular.\\~\\
Sean $\mathbb{A}:$ $a\rightarrow A\left(a\right)$ y $\mathbb{B}:$ $a\rightarrow B\left(a\right)$ representaciones de $G$. Si exite una matriz no singular $P$ tal que \\~\\
$B\left(a\right)=\qquad P^{-1}A\left(a\right)P$, para todo $a \in G$,\\~\\
diremos que $\mathbb{A}$ y $\mathbb{B}$ son equivalentes. Representaciones equivalentes son denotadas por $\mathbb{A} \sim \mathbb{B}$. La relación $\sim$ define una clase de equivalencia de representaciones de $G$.\\~\\
\textbf{Ejemplo 1.3.} Sea $S_{n}$ el grupo simétrico de grado $n$. Para un elemento\\~\\
$\qquad 
\sigma = 
\begin{pmatrix}
1 & 2 & \cdots  & n\\ 
s_{1} & s_{2} & \cdots & s_{n}
\end{pmatrix} 
\in S_{n}$\\~\\
sea $A\left(\sigma\right)$ la matriz cuyo $i$-ésimo renglón es $\left(0,...,0,1,0,...,0\right)$ con 1 en el $s_{i}$-ésimo lugar:\\~\\
$\qquad
A\left(\sigma\right) = \left(\alpha_{ij}\left(\sigma\right)\right) 
\qquad
\left(i,j=1,2,...,n\right)
$
con
\begin{equation*}
         \alpha_{ij}\left(\sigma\right) = \left\{
	       \begin{array}{ll}
		 1      & \mathrm{si\ } j = s_{i} \\
		 0      & \mathrm{otro\ caso\ } 
	       \end{array}
	     \right.
\end{equation*}
El mapeo $\sigma \rightarrow A\left(\sigma\right)$ es una representación fiel de $S_{n}$.\\~\\
\textbf{Ejemplo 1.4.} Sea $G$ un grupo finito que consiste de los elementos $a_{1},a_{2},...,a_{n}$  y sea $S^{G}$ el grupo simétrico en $G$. El mapeo lleva cada elemento de $a \in G$ a la permutación\\~\\
$\qquad 
\begin{pmatrix}
a_{1} & a_{2} & \cdots  & a_{n}\\ 
a_{1}a & a_{2}a & \cdots & a_{n}a
\end{pmatrix} 
\in S_{n}^{G}$\\~\\
es un homomorfismo biyectivo de $G$ a $S^{G}$. A la permutación anterior, se le asocia la matriz\\~\\
$A\left(a\right)=\left(\alpha_{ij}\left(a\right)\right)$\\~\\
con\\~\\
\begin{equation*}
         \alpha_{ij}\left(a\right) = \left\{
	       \begin{array}{ll}
		 1      & \mathrm{si\ } a_{i}a = a_{j} \\
		 0      & \mathrm{otro\ caso\ } 
	       \end{array}
	     \right.
\end{equation*}
\\~\\
como en el ejemplo 1.3. Entonces el mapeo $a \rightarrow A\left(\sigma\right)$ convierte una representación fiel de $G$. Ésta representación es llamada represetación regular derecha de $G$. Sea $\Delta\left(a\right)$\\~\\
\begin{equation*}
         \alpha_{ij}\left(a\right) = \left\{
	       \begin{array}{ll}
		 1      & \mathrm{si\ } a = 1 \\
		 0      & \mathrm{otro\ caso\ } 
	       \end{array}
	     \right.
\end{equation*}
\\~\\
entonces\\~\\
$\qquad 
A\left(a\right) = 
\begin{pmatrix}
\delta\left(a_{1}aa_{1}^{-1}\right) & \delta\left(a_{1}aa_{2}^{-1}\right) & \cdots  & \delta\left(a_{1}aa_{n}^{-1}\right)\\
\delta\left(a_{2}aa_{1}^{-1}\right) & \delta\left(a_{2}aa_{2}^{-1}\right) & \cdots  & \delta\left(a_{2}aa_{n}^{-1}\right)\\ 
\cdots & \cdots & \cdots & \cdots\\
\delta\left(a_{n}aa_{1}^{-1}\right) & \delta\left(a_{n}aa_{2}^{-1}\right) & \cdots  & \delta\left(a_{n}aa_{n}^{-1}\right)
\end{pmatrix} 
$\\~\\
Si a $\neq$ 1, cada entrada sobre la diagonal es cero.\\~\\
La representación regular izquierda de $G$ es definida similarmente usando el homomorfismo\\~\\
$\qquad 
\begin{pmatrix}
a_{1} & a_{2} & \cdots  & a_{n}\\ 
aa_{1} & aa_{2} & \cdots & aa_{n}
\end{pmatrix} $\\~\\
concretamente\\~\\
$\qquad 
A\left(a\right) = 
\begin{pmatrix}
\delta\left(a_{1}^{-1}aa_{1}\right) & \delta\left(a_{1}^{-1}aa_{2}\right) & \cdots  & \delta\left(a_{1}^{-1}aa_{n}\right)\\
\delta\left(a_{2}^{-1}aa_{1}\right) & \delta\left(a_{2}^{-1}aa_{2}\right) & \cdots  & \delta\left(a_{2}^{-1}aa_{n}\right)\\ 
\cdots & \cdots & \cdots & \cdots\\
\delta\left(a_{n}^{-1}aa_{1}\right) & \delta\left(a_{n}^{-1}aa_{2}\right) & \cdots  & \delta\left(a_{n}^{-1}aa_{n}\right)
\end{pmatrix} 
$\\~\\
Sea $\phi:a \rightarrow \phi\left(a\right)$ un homomorfismo de $G$ en $S_{n}$ (es decir, una permutación de de $G$). Expresando la permutación $\phi\left(a\right)$ por la matriz $A\left(a\right)$ como en el ejemplo 1.3, se obtiene una matriz representación $a \rightarrow A\left(a\right)$.\\~\\
Sea $\mathbb{A}: a \rightarrow A\left(a\right)$ una representación de grado $n$. Se dice que $\mathbb{A}$ es reducible si existe una matriz no singular, tal que \\~\\
$\qquad 
P^{-1}A\left(a\right)P=
\begin{pmatrix}
B\left(a\right) & 0 \\
D\left(a\right) & C\left(a\right)
\end{pmatrix} $, para todo $a \in G$,\\~\\
donde $B\left(a\right)$, $C\left(a\right)$ son matrices cuadradas de grado $r$, $s$ con $r \geq 1$, $s \geq 1$, $r+s=n$. Se observa que las representaciones \\~\\
$\qquad 
a \rightarrow A^{'}\left(a\right)=
\begin{pmatrix}
B\left(a\right) & 0 \\
D\left(a\right) & C\left(a\right)
\end{pmatrix}$\\~\\
y\\~\\
$\qquad 
a \rightarrow A^{''}\left(a\right)=
\begin{pmatrix}
C\left(a\right) & D\left(a\right) \\
0 & C\left(a\right)
\end{pmatrix}$\\~\\
son equivalentes, porque $Q^{-1}A^{'}\left(a\right)Q=A^{''}\left(a\right)$, con\\~\\
$\qquad 
Q=
\begin{pmatrix}
0 & \mathrm{I_{R}} \\ 
\mathrm{I_{S}} & 0
\end{pmatrix} \qquad \mathrm{(I_{r},I_{s}\ son\ las\ matrices\ identidad\ de\ grado\ r,s).\ }$\\~\\
Se dice que $\mathbb{A}$ es irreducible si no es reducible. En el ejemplo 1.3, el mapeo $a \rightarrow B\left(a\right)$ y $a \rightarrow C\left(a\right)$ convierten representaciones de grado $r,s$, respectivamente.\\~\\
Dada una representación de $G$, $\mathbb{A}: a \rightarrow A\left(a\right)$, y $\mathbb{B}: a \rightarrow B\left(a\right)$, con grado $n$, $m$, respectivamente, el mapeo.\\~\\
$\qquad 
Q=
\begin{pmatrix}
A\left(a\right) & 0 \\ 
0 & B\left(a\right)
\end{pmatrix}, \qquad \mathrm{para\ todo\ } a \in G$\\~\\
convierte en una representación de $G$ de grado $n+m$. Esta representación es llamada la suma directa de $\mathbb{A}$ y $\mathbb{B}$, y es denotada por $\mathbb{A}\oplus\mathbb{B}$.\\~\\
Una representación $\mathbb{A}: a \rightarrow A\left(a\right)$ de $G$ se dice completamente reducible si $\mathbb{A}$ es equivalente a la suma directa de algunas representaciones irreducibles, es decir, existe una matriz no singular $P$, tal que\\~\\
$\qquad 
P^{-1}A\left(a\right)P=
\begin{pmatrix}
\mathrm{F_{1}}\left(a\right) & & & & & 0\\
 & \mathrm{F_{2}}\left(a\right) & & & & \\
 & & . & & & \\
 & & & . & & \\
 & & & & . & \\
0 & & & & & \mathrm{F_{r}}\left(a\right)
\end{pmatrix}$,\\~\\
donde cada $\mathbb{F_{i}}: a \rightarrow F_{i}\left(a\right)$ $\left(i=1,2,...,r\right)$ es una representación irreducible de $G$.\\~\\

\subsubsection{Representación por matrices unitarias, y representaciones de completamente reducibles de grupos finitos}
Una representación $\mathbb{A}: a \rightarrow A\left(a\right)$ de $G$  se dice unitaria si $A\left(a\right)$ es una matriz unitaria para todo $a \in G$, lo cual significa que $\overline{A\left(a\right)}^{t}A\left(a\right)=\mathrm{I}$. Aquí $\overline{A\left(a\right)}^{t}$ denota la transpuesta de $\overline{A}=\left(\alpha_{ij}\right)$, donde $A=\left(\alpha_{ij}\right)$, y $\overline{\left(\alpha_{ij}\right)}$ es el complejo conjugado de $\left(\alpha_{ij}\right)$. Se pretende mostrar que cada representación de un grupo finito es equivalente a una representación unitaria y es completamente reducible.\\~\\
Una matriz se dice hermitiana si $\overline{A^{t}}=A$, y positiva definida si $\overline{x}^{t}Ax>0$ para todo vector columna $x$ (distinto de cero).\\~\\
\textbf{Lema 2.1.} Para cualquier matriz no singular $A$, $\overline{A\left(a\right)}^{t}A$ es una matriz hermitiana definida positiva. La suma de matrices hermitianas definidas positivas, también es hermitiana y definida positiva.\\~\\
\textbf{Lema 2.2.} Para cualquier matriz hermitiana definida positiva $A$, existe una matriz triangular superior no singular $C$ tal que $\overline{C}^{t}AC=\mathrm{I}$.\\~\\
Lo anterior es cierto, ya que, sea $A\left(\alpha_{ij}\right) \mathrm{con\ } \left(i,j=1,2,...,n\right)$. Entonces $\alpha_{ji}=\overline{\alpha_{ij}} \quad \mathrm{con\ } \left(i,j=1,2,...,n\right)$, y $\left(\alpha_{ii}\right)>0 \quad \mathrm{para\ } \left(i=1,2,...,n\right)$.\\~\\
Sea\\~\\
$\qquad 
A=
\begin{pmatrix}
\alpha & a \\ 
\overline{a}^{t} & \mathrm{B}
\end{pmatrix}, \quad \left(\alpha=\alpha_{11}>0,
a=\left(\alpha_{12},\alpha_{13},...,\alpha_{1n}\right),
\mathrm{B}=\left(\alpha_{ij}\right) \quad \left(i,j=2,...,n\right) \right) $\\~\\
sea\\~\\
$\qquad 
C_{1}=
\begin{pmatrix}
\frac{1}{\sqrt{\alpha}} & -\frac{1}{\alpha} \\ 
0 & \mathrm{I}
\end{pmatrix}$\\~\\
entonces, \\~\\
$\qquad 
\overline{C_{1}}^{t}AC_{1} =
\begin{pmatrix}
1 & 0 \\ 
0 & -\frac{1}{\alpha}\overline{a}^{t}a+\mathrm{B}
\end{pmatrix}$\\~\\
y $-\frac{1}{\alpha}\overline{a}^{t}a+\mathrm{B}$ es una matriz hermitiana definida positiva. Y la prueba se sigue usando inducción el grado de $A$ veces.\\~\\
\textbf{Teorema 2.3.} Sea $G$ un grupo finito. Para una representación $\mathbb{B}: a\rightarrow F\left(a\right)$ de $G$, entonces exite una matriz triangular superior no singular $C$, tal que $C_{-1}F\left(a\right)C$ es una matriz unitaria para todo $a \in G$.\\~\\
Sea\\~\\
\begin{equation*}
A=\sum_{b \in G} \overline{F\left(b\right)}^{t}F\left(b\right)
\end{equation*}
\\~\\
Entonces $A$ es una matriz hermitiana definida positiva por el Lemma 2.1. Entonces existe una matriz triangular no singular $C$, tal que\\~\\
$\qquad \overline{C}^{t}AC= \mathrm{I}$\\~\\
$\qquad A=\left(\overline{C}^{t}\right)C^{-1}$.\\
Entonces
\begin{equation*}
\overline{F\left(a\right)}^{t}AF\left(a\right)=\sum_{b \in G} \overline{F\left(ba\right)}^{t}F\left(ba\right)=A
\end{equation*},
y se obtiene
\begin{equation*}
\overline{F\left(a\right)}^{t}(\overline{C}^{t})^{-1}C^{-1}F\left(a\right)=(\overline{C}^{t})^{-1}C^{-1}
\end{equation*},
es decir\\~\\
\begin{equation*}
\overline{(C_{-1}F(a)C)}^{t}(C_{-1}F(a)^{t}C)=\mathrm{I}
\end{equation*}
y $C_{-1}F(a)^{t}C$ es una matriz unitaria.\\~\\
\textbf{Teorema 2.4.} Una representación de un grupo finito es completamente reducible.\\~\\
Sea $\mathbb{A}:$ $a\rightarrow A\left(a\right)$ una representación de un grupo finito de $G$ y sea $A(a)$ descompuesta como\\~\\
$\qquad
A(a)=
\begin{pmatrix}
A_{1}(a) & * \\ 
0 & A_{2}(a)
\end{pmatrix}$.\\~\\
Por el teorema anterior, existe una matriz triangular no superior $C$ tal que $C^{-1}A(a)C$ es una matriz unitaria. Sea $U(a)=C^{-1}A(a)C$. Como $C$ es una matriz triangular superior, $U(a)$ se descompone como\\~\\
$\qquad
U(a)=
\begin{pmatrix}
U_{1}(a) & V(a) \\ 
0 & U_{2}(a)
\end{pmatrix}$.\\~\\
Como $\overline{U(a)}^{t}=U(a)^{-1}=U(a^{-1})$, se obtiene\\~\\
$\qquad
\begin{pmatrix}
\overline{U_{1}(a)} & 0 \\ 
\overline{V(a)}^{t} & \overline{U_{2}(a)}^{t}
\end{pmatrix}
=
\begin{pmatrix}
U_{1}(a^{-1}) & V(a^{-1}) \\ 
0 & U_{2}(a^{-1})
\end{pmatrix}$.\\~\\
\textbf{Lema 3.1.} (Lema de Schur) Sea $\mathbb{A}:$ $a\rightarrow A\left(a\right)$ y $\mathbb{B}:$ $a\rightarrow B\left(a\right)$ representaciones irreducibles de un grupo $G$ con grados $m$ y $n$ respectivamente. Sea $P$ una matriz de $m$x$n$ con la propierdad de que\\~\\
$A(a)P=PB(a)$, para todo $a \in G$.
\backmatter

\bibliographystyle{plain}
\bibliography{labiblio}

\printindex


\end{document}
